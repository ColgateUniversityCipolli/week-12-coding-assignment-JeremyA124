\documentclass{article}\usepackage[]{graphicx}\usepackage[]{xcolor}
% maxwidth is the original width if it is less than linewidth
% otherwise use linewidth (to make sure the graphics do not exceed the margin)
\makeatletter
\def\maxwidth{ %
  \ifdim\Gin@nat@width>\linewidth
    \linewidth
  \else
    \Gin@nat@width
  \fi
}
\makeatother

\definecolor{fgcolor}{rgb}{0.345, 0.345, 0.345}
\newcommand{\hlnum}[1]{\textcolor[rgb]{0.686,0.059,0.569}{#1}}%
\newcommand{\hlsng}[1]{\textcolor[rgb]{0.192,0.494,0.8}{#1}}%
\newcommand{\hlcom}[1]{\textcolor[rgb]{0.678,0.584,0.686}{\textit{#1}}}%
\newcommand{\hlopt}[1]{\textcolor[rgb]{0,0,0}{#1}}%
\newcommand{\hldef}[1]{\textcolor[rgb]{0.345,0.345,0.345}{#1}}%
\newcommand{\hlkwa}[1]{\textcolor[rgb]{0.161,0.373,0.58}{\textbf{#1}}}%
\newcommand{\hlkwb}[1]{\textcolor[rgb]{0.69,0.353,0.396}{#1}}%
\newcommand{\hlkwc}[1]{\textcolor[rgb]{0.333,0.667,0.333}{#1}}%
\newcommand{\hlkwd}[1]{\textcolor[rgb]{0.737,0.353,0.396}{\textbf{#1}}}%
\let\hlipl\hlkwb

\usepackage{framed}
\makeatletter
\newenvironment{kframe}{%
 \def\at@end@of@kframe{}%
 \ifinner\ifhmode%
  \def\at@end@of@kframe{\end{minipage}}%
  \begin{minipage}{\columnwidth}%
 \fi\fi%
 \def\FrameCommand##1{\hskip\@totalleftmargin \hskip-\fboxsep
 \colorbox{shadecolor}{##1}\hskip-\fboxsep
     % There is no \\@totalrightmargin, so:
     \hskip-\linewidth \hskip-\@totalleftmargin \hskip\columnwidth}%
 \MakeFramed {\advance\hsize-\width
   \@totalleftmargin\z@ \linewidth\hsize
   \@setminipage}}%
 {\par\unskip\endMakeFramed%
 \at@end@of@kframe}
\makeatother

\definecolor{shadecolor}{rgb}{.97, .97, .97}
\definecolor{messagecolor}{rgb}{0, 0, 0}
\definecolor{warningcolor}{rgb}{1, 0, 1}
\definecolor{errorcolor}{rgb}{1, 0, 0}
\newenvironment{knitrout}{}{} % an empty environment to be redefined in TeX

\usepackage{alltt}
\usepackage[margin=1.0in]{geometry} % To set margins
\usepackage{amsmath}  % This allows me to use the align functionality.
                      % If you find yourself trying to replicate
                      % something you found online, ensure you're
                      % loading the necessary packages!
\usepackage{amsfonts} % Math font
\usepackage{fancyvrb}
\usepackage{hyperref} % For including hyperlinks
\usepackage[shortlabels]{enumitem}% For enumerated lists with labels specified
                                  % We had to run tlmgr_install("enumitem") in R
\usepackage{float}    % For telling R where to put a table/figure
\usepackage{natbib}        %For the bibliography
\bibliographystyle{apalike}%For the bibliography
\IfFileExists{upquote.sty}{\usepackage{upquote}}{}
\begin{document}


\begin{enumerate}
%%%%%%%%%%%%%%%%%%%%%%%%%%%%%%%%%%%%%%%%%%%%%%%%%%%%%%%%%%%%%%%%%%%%%%%%%%%%%%%%
%%%%%%%%%%%%%%%%%%%%%%%%%%%%%%%%%%%%%%%%%%%%%%%%%%%%%%%%%%%%%%%%%%%%%%%%%%%%%%%%
% Question 1
%%%%%%%%%%%%%%%%%%%%%%%%%%%%%%%%%%%%%%%%%%%%%%%%%%%%%%%%%%%%%%%%%%%%%%%%%%%%%%%%
%%%%%%%%%%%%%%%%%%%%%%%%%%%%%%%%%%%%%%%%%%%%%%%%%%%%%%%%%%%%%%%%%%%%%%%%%%%%%%%%
\item A group of researchers is running an experiment over the course of 30 months, 
with a single observation collected at the end of each month. Let $X_1, ..., X_{30}$
denote the observations for each month. From prior studies, the researchers know that
\[X_i \sim f_X(x),\]
but the mean $\mu_X$ is unknown, and they wish to conduct the following test
\begin{align*}
H_0&: \mu_X = 0\\
H_a&: \mu_X > 0.
\end{align*}
At month $k$, they have accumulated data $X_1, ..., X_k$ and they have the 
$t$-statistic
\[T_k = \frac{\bar{X} - 0}{S_k/\sqrt{n}}.\]
The initial plan was to test the hypotheses after all data was collected (at the 
end of month 30), at level $\alpha=0.05$. However, conducting the experiment is 
expensive, so the researchers want to ``peek" at the data at the end of month 20 
to see if they can stop it early. That is, the researchers propose to check 
whether $t_{20}$ provides statistically discernible support for the alternative. 
If it does, they will stop the experiment early and report support for the 
researcher's alternative hypothesis. If it does not, they will continue to month 
30 and test whether $t_{30}$ provides statistically discernible support for the
alternative.

\begin{enumerate}
  \item What values of $t_{20}$ provide statistically discernible support for the
  alternative hypothesis? \\
  \textbf{Solution:} Below, we find the value of $t_{20}$ that provides statistically discernible support for the
  alternative hypothesis?
\begin{knitrout}\scriptsize
\definecolor{shadecolor}{rgb}{0.969, 0.969, 0.969}\color{fgcolor}\begin{kframe}
\begin{alltt}
\hldef{n} \hlkwb{<-} \hlnum{20}
\hlkwd{qt}\hldef{(}\hlnum{0.95}\hldef{,} \hlkwc{df} \hldef{= n} \hlopt{-} \hlnum{1}\hldef{)} \hlcom{#find the value for t for 20 observations}
\end{alltt}
\begin{verbatim}
## [1] 1.729133
\end{verbatim}
\end{kframe}
\end{knitrout}
  
  \item What values of $t_{30}$ provide statistically discernible support for the
  alternative hypothesis? \\
  \textbf{Solution:} Below, we find the value of $t_{30}$ that provides statistically discernible support for the
\begin{knitrout}\scriptsize
\definecolor{shadecolor}{rgb}{0.969, 0.969, 0.969}\color{fgcolor}\begin{kframe}
\begin{alltt}
\hldef{n} \hlkwb{<-} \hlnum{30}
\hlkwd{qt}\hldef{(}\hlnum{0.95}\hldef{,} \hlkwc{df} \hldef{= n} \hlopt{-} \hlnum{1}\hldef{)} \hlcom{#find the value for t for 30 observations}
\end{alltt}
\begin{verbatim}
## [1] 1.699127
\end{verbatim}
\end{kframe}
\end{knitrout}
  
  \item Suppose $f_X(x)$ is a Laplace distribution with $a=0$ and $b=4.0$.
  Conduct a simulation study to assess the Type I error rate of this approach.\\
  \textbf{Note:} You can use the \texttt{rlaplace()} function from the \texttt{VGAM}
  package for \texttt{R} \citep{VGAM}. \\
  \textbf{Solution:} Below, we find the Type I error rate of this approach.
\begin{knitrout}\scriptsize
\definecolor{shadecolor}{rgb}{0.969, 0.969, 0.969}\color{fgcolor}\begin{kframe}
\begin{alltt}
\hldef{R} \hlkwb{<-} \hlnum{1000} \hlcom{#resample size}
\hldef{hypo.check} \hlkwb{<-} \hlkwd{tibble}\hldef{(}  \hlcom{#tibble to store data}
  \hlkwc{reject} \hldef{=} \hlkwd{rep}\hldef{(}\hlnum{NA}\hldef{, R)}
\hldef{)}

\hlkwa{for} \hldef{(i} \hlkwa{in} \hlnum{1}\hlopt{:}\hldef{R)\{}
  \hldef{sim.data.30} \hlkwb{<-} \hlkwd{rlaplace}\hldef{(}\hlkwc{n} \hldef{=} \hlnum{30}\hldef{,} \hlkwc{location} \hldef{=} \hlnum{0}\hldef{,} \hlkwc{scale} \hldef{=} \hlnum{4}\hldef{)} \hlcom{#simulate rlapace data for n = 30}
  \hldef{sim.data.20} \hlkwb{<-} \hldef{sim.data.30[}\hlnum{1}\hlopt{:}\hlnum{20}\hldef{]} \hlcom{#take the first 20 observations}

  \hldef{p.value.20} \hlkwb{<-} \hlkwd{t.test}\hldef{(sim.data.20,} \hlcom{#grab the p-value for the first 20 observations}
                       \hlkwc{mu}\hldef{=} \hlnum{0}\hldef{,}
                       \hlkwc{alternative} \hldef{=} \hlsng{"greater"}\hldef{)}\hlopt{$}\hldef{p.value}
  \hldef{p.value.30} \hlkwb{<-} \hlkwd{t.test}\hldef{(sim.data.30,} \hlcom{#grab the p-value for the first 30 observations}
                       \hlkwc{mu}\hldef{=} \hlnum{0}\hldef{,}
                       \hlkwc{alternative} \hldef{=} \hlsng{"greater"}\hldef{)}\hlopt{$}\hldef{p.value}

  \hlcom{#Perform a hyphesis check for n = 20 and n = 30}
  \hldef{hypo.check}\hlopt{$}\hldef{reject[i]} \hlkwb{<-} \hlkwd{ifelse}\hldef{(p.value.20} \hlopt{<=} \hlnum{0.05}\hldef{,} \hlnum{1}\hldef{,}
                                 \hlkwd{ifelse}\hldef{(p.value.30} \hlopt{<=} \hlnum{0.05}\hldef{,} \hlnum{1}\hldef{,} \hlnum{0}\hldef{))}
\hldef{\}}

\hldef{type1.error.rate} \hlkwb{<-} \hlkwd{sum}\hldef{(hypo.check}\hlopt{$}\hldef{reject)}\hlopt{/}\hldef{R} \hlcom{#Calulcate the Type I error rate}
\hldef{type1.error.rate} \hlcom{#display}
\end{alltt}
\begin{verbatim}
## [1] 0.098
\end{verbatim}
\end{kframe}
\end{knitrout}
  
  \item \textbf{Optional Challenge:} Can you find a value of $\alpha<0.05$ that yields a 
  Type I error rate of 0.05?
  \textbf{Solution:} Below, we find the value of $\alpha<0.05$ that yields a 
  Type I error rate of 0.05.
\begin{knitrout}\scriptsize
\definecolor{shadecolor}{rgb}{0.969, 0.969, 0.969}\color{fgcolor}\begin{kframe}
\begin{alltt}
\hldef{solve} \hlkwb{<-} \hlkwa{function}\hldef{(}\hlkwc{x}\hldef{) \{} \hlcom{#Create a function to solve for f(x) = 0}
  \hldef{hypo.check} \hlkwb{<-} \hlkwd{tibble}\hldef{(} \hlcom{#tibble to store data}
    \hlkwc{reject} \hldef{=} \hlkwd{rep}\hldef{(}\hlnum{NA}\hldef{, R)}
  \hldef{)}

  \hlkwa{for} \hldef{(i} \hlkwa{in} \hlnum{1}\hlopt{:}\hldef{R)\{}
    \hldef{sim.data.30} \hlkwb{<-} \hlkwd{rlaplace}\hldef{(}\hlkwc{n} \hldef{=} \hlnum{30}\hldef{,} \hlkwc{location} \hldef{=} \hlnum{0}\hldef{,} \hlkwc{scale} \hldef{=} \hlnum{4}\hldef{)} \hlcom{#simulate rlapace data for n = 30}
    \hldef{sim.data.20} \hlkwb{<-} \hldef{sim.data.30[}\hlnum{1}\hlopt{:}\hlnum{20}\hldef{]} \hlcom{#take the first 20 observations}

    \hldef{p.value.20} \hlkwb{<-} \hlkwd{t.test}\hldef{(sim.data.20,} \hlcom{#grab the p-value for the first 20 observations}
                         \hlkwc{mu}\hldef{=} \hlnum{0}\hldef{,}
                         \hlkwc{alternative} \hldef{=} \hlsng{"greater"}\hldef{)}\hlopt{$}\hldef{p.value}
    \hldef{p.value.30} \hlkwb{<-} \hlkwd{t.test}\hldef{(sim.data.30,} \hlcom{#grab the p-value for the first 30 observations}
                         \hlkwc{mu}\hldef{=} \hlnum{0}\hldef{,}
                         \hlkwc{alternative} \hldef{=} \hlsng{"greater"}\hldef{)}\hlopt{$}\hldef{p.value}

    \hlcom{#Perform a hyphesis check for n = 20 and n = 30}
    \hldef{hypo.check}\hlopt{$}\hldef{reject[i]} \hlkwb{<-} \hlkwd{ifelse}\hldef{(p.value.20} \hlopt{<=} \hldef{x,} \hlnum{1}\hldef{,}
                                   \hlkwd{ifelse}\hldef{(p.value.30} \hlopt{<=} \hldef{x,} \hlnum{1}\hldef{,} \hlnum{0}\hldef{))}
  \hldef{\}}

  \hlkwd{return}\hldef{(}\hlnum{50}\hlopt{-}\hlkwd{sum}\hldef{(hypo.check))} \hlcom{#return the solution (Should be 0)}
\hldef{\}}

\hldef{alpha} \hlkwb{<-} \hlkwd{uniroot}\hldef{(solve,} \hlkwc{lower} \hldef{=} \hlnum{0}\hldef{,} \hlkwc{upper} \hldef{=} \hlnum{0.05}\hldef{)}\hlopt{$}\hldef{root} \hlcom{#solve for alpha}
\hldef{alpha} \hlcom{#display}
\end{alltt}
\begin{verbatim}
## [1] 0.03606428
\end{verbatim}
\end{kframe}
\end{knitrout}
  
\end{enumerate}
%%%%%%%%%%%%%%%%%%%%%%%%%%%%%%%%%%%%%%%%%%%%%%%%%%%%%%%%%%%%%%%%%%%%%%%%%%%%%%%%
%%%%%%%%%%%%%%%%%%%%%%%%%%%%%%%%%%%%%%%%%%%%%%%%%%%%%%%%%%%%%%%%%%%%%%%%%%%%%%%%
% Question 2
%%%%%%%%%%%%%%%%%%%%%%%%%%%%%%%%%%%%%%%%%%%%%%%%%%%%%%%%%%%%%%%%%%%%%%%%%%%%%%%%
%%%%%%%%%%%%%%%%%%%%%%%%%%%%%%%%%%%%%%%%%%%%%%%%%%%%%%%%%%%%%%%%%%%%%%%%%%%%%%%%
  \item Perform a simulation study to assess the robustness of the $T$ test. 
  Specifically, generate samples of size $n=15$ from the Beta(10,2), Beta(2,10), 
  and Beta(10,10) distributions and conduct the following hypothesis tests against 
  the actual mean for each case (e.g., $\frac{10}{10+2}$, $\frac{2}{10+2}$, and 
  $\frac{10}{10+10}$). 
  \begin{enumerate}
    \item What proportion of the time do we make an error of Type I for a
    left-tailed test? \\
    \textbf{Solution:} Below, we find the proportion of the times we make a Type I error for the Beta(10,2), Beta(2,10), and Beta(10,10) distributions with a left-tailed test.
\begin{knitrout}\scriptsize
\definecolor{shadecolor}{rgb}{0.969, 0.969, 0.969}\color{fgcolor}\begin{kframe}
\begin{alltt}
\hldef{hypo.check.less} \hlkwb{<-} \hlkwd{tibble}\hldef{(} \hlcom{#Tibble to store our data}
  \hlkwc{reject.10.2} \hldef{=} \hlkwd{rep}\hldef{(}\hlnum{NA}\hldef{, R),}
  \hlkwc{reject.2.10} \hldef{=} \hlkwd{rep}\hldef{(}\hlnum{NA}\hldef{, R),}
  \hlkwc{reject.10.10} \hldef{=} \hlkwd{rep}\hldef{(}\hlnum{NA}\hldef{, R)}
\hldef{)}

\hlkwa{for} \hldef{(i} \hlkwa{in} \hlnum{1}\hlopt{:}\hldef{R)\{}
  \hldef{sample.beta.10.2} \hlkwb{<-} \hlkwd{rbeta}\hldef{(}\hlkwc{n} \hldef{=} \hlnum{15}\hldef{,} \hlkwc{shape1} \hldef{=} \hlnum{10}\hldef{,} \hlkwc{shape2} \hldef{=} \hlnum{2}\hldef{)} \hlcom{#Generate some samples of beta (10,2)}
  \hldef{sample.beta.2.10} \hlkwb{<-} \hlkwd{rbeta}\hldef{(}\hlkwc{n} \hldef{=} \hlnum{15}\hldef{,} \hlkwc{shape1} \hldef{=} \hlnum{2}\hldef{,} \hlkwc{shape2} \hldef{=} \hlnum{10}\hldef{)} \hlcom{#Generate some samples of beta (2,10)}
  \hldef{sample.beta.10.10} \hlkwb{<-} \hlkwd{rbeta}\hldef{(}\hlkwc{n}\hldef{=} \hlnum{15}\hldef{,} \hlkwc{shape1} \hldef{=} \hlnum{10}\hldef{,} \hlkwc{shape2} \hldef{=} \hlnum{10}\hldef{)} \hlcom{#Generate some samples of beta (10,10)}

  \hldef{p.value.1} \hlkwb{<-} \hlkwd{t.test}\hldef{(sample.beta.10.2,}           \hlcom{#Perform t-test and grab p-values for all samples}
                      \hlkwc{mu} \hldef{=} \hlnum{10}\hlopt{/}\hlnum{12}\hldef{,}
                      \hlkwc{alternative} \hldef{=} \hlsng{"less"}\hldef{)}\hlopt{$}\hldef{p.value}

  \hldef{p.value.2} \hlkwb{<-} \hlkwd{t.test}\hldef{(sample.beta.2.10,}
                      \hlkwc{mu} \hldef{=} \hlnum{2}\hlopt{/}\hlnum{12}\hldef{,}
                      \hlkwc{alternative} \hldef{=} \hlsng{"less"}\hldef{)}\hlopt{$}\hldef{p.value}

  \hldef{p.value.3} \hlkwb{<-} \hlkwd{t.test}\hldef{(sample.beta.10.10,}
                      \hlkwc{mu} \hldef{=} \hlnum{10}\hlopt{/}\hlnum{20}\hldef{,}
                      \hlkwc{alternative} \hldef{=} \hlsng{"less"}\hldef{)}\hlopt{$}\hldef{p.value}

  \hlcom{#Do hypothesis comparison and determine whether p-value < 0.05 (alpha)}
  \hldef{hypo.check.less}\hlopt{$}\hldef{reject.10.2[i]} \hlkwb{<-} \hlkwd{ifelse}\hldef{(p.value.1} \hlopt{<=} \hlnum{0.05}\hldef{,} \hlnum{1}\hldef{,} \hlnum{0}\hldef{)}
  \hldef{hypo.check.less}\hlopt{$}\hldef{reject.2.10[i]} \hlkwb{<-} \hlkwd{ifelse}\hldef{(p.value.2} \hlopt{<=} \hlnum{0.05}\hldef{,} \hlnum{1}\hldef{,} \hlnum{0}\hldef{)}
  \hldef{hypo.check.less}\hlopt{$}\hldef{reject.10.10[i]} \hlkwb{<-} \hlkwd{ifelse}\hldef{(p.value.3} \hlopt{<=} \hlnum{0.05}\hldef{,} \hlnum{1}\hldef{,} \hlnum{0}\hldef{)}
\hldef{\}}

\hldef{type1.error.rate.10.2} \hlkwb{<-} \hlkwd{sum}\hldef{(hypo.check.less}\hlopt{$}\hldef{reject.10.2)}\hlopt{/}\hldef{R} \hlcom{#Calculates type 1 error proportions}
\hldef{type1.error.rate.2.10} \hlkwb{<-} \hlkwd{sum}\hldef{(hypo.check.less}\hlopt{$}\hldef{reject.2.10)}\hlopt{/}\hldef{R}
\hldef{type1.error.rate.10.10} \hlkwb{<-} \hlkwd{sum}\hldef{(hypo.check.less}\hlopt{$}\hldef{reject.10.10)}\hlopt{/}\hldef{R}

\hldef{type1.error.rate.10.2}
\end{alltt}
\begin{verbatim}
## [1] 0.022
\end{verbatim}
\begin{alltt}
\hldef{type1.error.rate.2.10}
\end{alltt}
\begin{verbatim}
## [1] 0.07
\end{verbatim}
\begin{alltt}
\hldef{type1.error.rate.10.10}
\end{alltt}
\begin{verbatim}
## [1] 0.052
\end{verbatim}
\end{kframe}
\end{knitrout}
    
    \item What proportion of the time do we make an error of Type I for a
    right-tailed test? \\
    \textbf{Solution:} Below, we find the proportion of the times we make a Type I error for the Beta(10,2), Beta(2,10), and Beta(10,10) distributions with a right-tailed test.
\begin{knitrout}\scriptsize
\definecolor{shadecolor}{rgb}{0.969, 0.969, 0.969}\color{fgcolor}\begin{kframe}
\begin{alltt}
\hldef{hypo.check.greater} \hlkwb{<-} \hlkwd{tibble}\hldef{(} \hlcom{#Tibble to store our data}
  \hlkwc{reject.10.2} \hldef{=} \hlkwd{rep}\hldef{(}\hlnum{NA}\hldef{, R),}
  \hlkwc{reject.2.10} \hldef{=} \hlkwd{rep}\hldef{(}\hlnum{NA}\hldef{, R),}
  \hlkwc{reject.10.10} \hldef{=} \hlkwd{rep}\hldef{(}\hlnum{NA}\hldef{, R)}
\hldef{)}

\hlkwa{for} \hldef{(i} \hlkwa{in} \hlnum{1}\hlopt{:}\hldef{R)\{}
  \hldef{sample.beta.10.2} \hlkwb{<-} \hlkwd{rbeta}\hldef{(}\hlkwc{n} \hldef{=} \hlnum{15}\hldef{,} \hlkwc{shape1} \hldef{=} \hlnum{10}\hldef{,} \hlkwc{shape2} \hldef{=} \hlnum{2}\hldef{)} \hlcom{#Generate some samples of beta (10,2)}
  \hldef{sample.beta.2.10} \hlkwb{<-} \hlkwd{rbeta}\hldef{(}\hlkwc{n} \hldef{=} \hlnum{15}\hldef{,} \hlkwc{shape1} \hldef{=} \hlnum{2}\hldef{,} \hlkwc{shape2} \hldef{=} \hlnum{10}\hldef{)} \hlcom{#Generate some samples of beta (2,10)}
  \hldef{sample.beta.10.10} \hlkwb{<-} \hlkwd{rbeta}\hldef{(}\hlkwc{n}\hldef{=} \hlnum{15}\hldef{,} \hlkwc{shape1} \hldef{=} \hlnum{10}\hldef{,} \hlkwc{shape2} \hldef{=} \hlnum{10}\hldef{)} \hlcom{#Generate some samples of beta (10,10)}

  \hldef{p.value.1} \hlkwb{<-} \hlkwd{t.test}\hldef{(sample.beta.10.2,}           \hlcom{#Perform t-test and grab p-values for all samples}
                      \hlkwc{mu} \hldef{=} \hlnum{10}\hlopt{/}\hlnum{12}\hldef{,}
                      \hlkwc{alternative} \hldef{=} \hlsng{"greater"}\hldef{)}\hlopt{$}\hldef{p.value}

  \hldef{p.value.2} \hlkwb{<-} \hlkwd{t.test}\hldef{(sample.beta.2.10,}
                      \hlkwc{mu} \hldef{=} \hlnum{2}\hlopt{/}\hlnum{12}\hldef{,}
                      \hlkwc{alternative} \hldef{=} \hlsng{"greater"}\hldef{)}\hlopt{$}\hldef{p.value}

  \hldef{p.value.3} \hlkwb{<-} \hlkwd{t.test}\hldef{(sample.beta.10.10,}
                      \hlkwc{mu} \hldef{=} \hlnum{10}\hlopt{/}\hlnum{20}\hldef{,}
                      \hlkwc{alternative} \hldef{=} \hlsng{"greater"}\hldef{)}\hlopt{$}\hldef{p.value}

  \hlcom{#Do hypothesis comparison and determine whether p-value < 0.05 (alpha)}
  \hldef{hypo.check.greater}\hlopt{$}\hldef{reject.10.2[i]} \hlkwb{<-} \hlkwd{ifelse}\hldef{(p.value.1} \hlopt{<=} \hlnum{0.05}\hldef{,} \hlnum{1}\hldef{,} \hlnum{0}\hldef{)}
  \hldef{hypo.check.greater}\hlopt{$}\hldef{reject.2.10[i]} \hlkwb{<-} \hlkwd{ifelse}\hldef{(p.value.2} \hlopt{<=} \hlnum{0.05}\hldef{,} \hlnum{1}\hldef{,} \hlnum{0}\hldef{)}
  \hldef{hypo.check.greater}\hlopt{$}\hldef{reject.10.10[i]} \hlkwb{<-} \hlkwd{ifelse}\hldef{(p.value.3} \hlopt{<=} \hlnum{0.05}\hldef{,} \hlnum{1}\hldef{,} \hlnum{0}\hldef{)}
\hldef{\}}

\hldef{type1.error.rate.10.2} \hlkwb{<-} \hlkwd{sum}\hldef{(hypo.check.greater}\hlopt{$}\hldef{reject.10.2)}\hlopt{/}\hldef{R} \hlcom{#Calculates type 1 error proportions}
\hldef{type1.error.rate.2.10} \hlkwb{<-} \hlkwd{sum}\hldef{(hypo.check.greater}\hlopt{$}\hldef{reject.2.10)}\hlopt{/}\hldef{R}
\hldef{type1.error.rate.10.10} \hlkwb{<-} \hlkwd{sum}\hldef{(hypo.check.greater}\hlopt{$}\hldef{reject.10.10)}\hlopt{/}\hldef{R}

\hldef{type1.error.rate.10.2}
\end{alltt}
\begin{verbatim}
## [1] 0.075
\end{verbatim}
\begin{alltt}
\hldef{type1.error.rate.2.10}
\end{alltt}
\begin{verbatim}
## [1] 0.021
\end{verbatim}
\begin{alltt}
\hldef{type1.error.rate.10.10}
\end{alltt}
\begin{verbatim}
## [1] 0.052
\end{verbatim}
\end{kframe}
\end{knitrout}
    
    \item What proportion of the time do we make an error of Type I for a
    two-tailed test? \\
    \textbf{Solution:} Below, we find the proportion of the times we make a Type I error for the Beta(10,2), Beta(2,10), and Beta(10,10) distributions with a two-tailed test.
\begin{knitrout}\scriptsize
\definecolor{shadecolor}{rgb}{0.969, 0.969, 0.969}\color{fgcolor}\begin{kframe}
\begin{alltt}
\hldef{hypo.check.two.sided} \hlkwb{<-} \hlkwd{tibble}\hldef{(} \hlcom{#Tibble to store our data}
  \hlkwc{reject.10.2} \hldef{=} \hlkwd{rep}\hldef{(}\hlnum{NA}\hldef{, R),}
  \hlkwc{reject.2.10} \hldef{=} \hlkwd{rep}\hldef{(}\hlnum{NA}\hldef{, R),}
  \hlkwc{reject.10.10} \hldef{=} \hlkwd{rep}\hldef{(}\hlnum{NA}\hldef{, R)}
\hldef{)}

\hlkwa{for} \hldef{(i} \hlkwa{in} \hlnum{1}\hlopt{:}\hldef{R)\{}
  \hldef{sample.beta.10.2} \hlkwb{<-} \hlkwd{rbeta}\hldef{(}\hlkwc{n} \hldef{=} \hlnum{15}\hldef{,} \hlkwc{shape1} \hldef{=} \hlnum{10}\hldef{,} \hlkwc{shape2} \hldef{=} \hlnum{2}\hldef{)} \hlcom{#Generate some samples of beta (10,2)}
  \hldef{sample.beta.2.10} \hlkwb{<-} \hlkwd{rbeta}\hldef{(}\hlkwc{n} \hldef{=} \hlnum{15}\hldef{,} \hlkwc{shape1} \hldef{=} \hlnum{2}\hldef{,} \hlkwc{shape2} \hldef{=} \hlnum{10}\hldef{)} \hlcom{#Generate some samples of beta (2,10)}
  \hldef{sample.beta.10.10} \hlkwb{<-} \hlkwd{rbeta}\hldef{(}\hlkwc{n}\hldef{=} \hlnum{15}\hldef{,} \hlkwc{shape1} \hldef{=} \hlnum{10}\hldef{,} \hlkwc{shape2} \hldef{=} \hlnum{10}\hldef{)} \hlcom{#Generate some samples of beta (10,10)}

  \hldef{p.value.1} \hlkwb{<-} \hlkwd{t.test}\hldef{(sample.beta.10.2,}           \hlcom{#Perform t-test and grab p-values for all samples}
                      \hlkwc{mu} \hldef{=} \hlnum{10}\hlopt{/}\hlnum{12}\hldef{,}
                      \hlkwc{alternative} \hldef{=} \hlsng{"two.sided"}\hldef{)}\hlopt{$}\hldef{p.value}

  \hldef{p.value.2} \hlkwb{<-} \hlkwd{t.test}\hldef{(sample.beta.2.10,}
                      \hlkwc{mu} \hldef{=} \hlnum{2}\hlopt{/}\hlnum{12}\hldef{,}
                      \hlkwc{alternative} \hldef{=} \hlsng{"two.sided"}\hldef{)}\hlopt{$}\hldef{p.value}

  \hldef{p.value.3} \hlkwb{<-} \hlkwd{t.test}\hldef{(sample.beta.10.10,}
                      \hlkwc{mu} \hldef{=} \hlnum{10}\hlopt{/}\hlnum{20}\hldef{,}
                      \hlkwc{alternative} \hldef{=} \hlsng{"two.sided"}\hldef{)}\hlopt{$}\hldef{p.value}

  \hlcom{#Do hypothesis comparison and determine whether p-value < 0.05 (alpha)}
  \hldef{hypo.check.two.sided}\hlopt{$}\hldef{reject.10.2[i]} \hlkwb{<-} \hlkwd{ifelse}\hldef{(p.value.1} \hlopt{<=} \hlnum{0.05}\hldef{,} \hlnum{1}\hldef{,} \hlnum{0}\hldef{)}
  \hldef{hypo.check.two.sided}\hlopt{$}\hldef{reject.2.10[i]} \hlkwb{<-} \hlkwd{ifelse}\hldef{(p.value.2} \hlopt{<=} \hlnum{0.05}\hldef{,} \hlnum{1}\hldef{,} \hlnum{0}\hldef{)}
  \hldef{hypo.check.two.sided}\hlopt{$}\hldef{reject.10.10[i]} \hlkwb{<-} \hlkwd{ifelse}\hldef{(p.value.3} \hlopt{<=} \hlnum{0.05}\hldef{,} \hlnum{1}\hldef{,} \hlnum{0}\hldef{)}
\hldef{\}}

\hlcom{#Calculates type 1 error proportions}
\hldef{type1.error.rate.10.2} \hlkwb{<-} \hlkwd{sum}\hldef{(hypo.check.two.sided}\hlopt{$}\hldef{reject.10.2)}\hlopt{/}\hldef{R}
\hldef{type1.error.rate.2.10} \hlkwb{<-} \hlkwd{sum}\hldef{(hypo.check.two.sided}\hlopt{$}\hldef{reject.2.10)}\hlopt{/}\hldef{R}
\hldef{type1.error.rate.10.10} \hlkwb{<-} \hlkwd{sum}\hldef{(hypo.check.two.sided}\hlopt{$}\hldef{reject.10.10)}\hlopt{/}\hldef{R}

\hldef{type1.error.rate.10.2}
\end{alltt}
\begin{verbatim}
## [1] 0.062
\end{verbatim}
\begin{alltt}
\hldef{type1.error.rate.2.10}
\end{alltt}
\begin{verbatim}
## [1] 0.061
\end{verbatim}
\begin{alltt}
\hldef{type1.error.rate.10.10}
\end{alltt}
\begin{verbatim}
## [1] 0.059
\end{verbatim}
\end{kframe}
\end{knitrout}
    
    \item How does skewness of the underlying population distribution effect
    Type I error across the test types? \\
    \textbf{Solution:} As seen by our Type I error proportions, Type I error can be inflated for deflated depending on the skewness of the data and the type of test we use. For the right skewed distributions (Beta(2,10)), Type I error is inflated with right-tailed tests and deflated with left-tailed tests. For the left-skewed distributions (Beta(10,2)), Type I error is inflated with left-tailed tests and deflated with right-tailed tests. With symmetric distributions (10,10), we get balanced Type I errors with both left and right tailed tests. However, we also get balanced Type I errors when we use a two-sided test for all three distributoon types.
  \end{enumerate}
%%%%%%%%%%%%%%%%%%%%%%%%%%%%%%%%%%%%%%%%%%%%%%%%%%%%%%%%%%%%%%%%%%%%%%%%%%%%%%%%
%%%%%%%%%%%%%%%%%%%%%%%%%%%%%%%%%%%%%%%%%%%%%%%%%%%%%%%%%%%%%%%%%%%%%%%%%%%%%%%%
% End Document
%%%%%%%%%%%%%%%%%%%%%%%%%%%%%%%%%%%%%%%%%%%%%%%%%%%%%%%%%%%%%%%%%%%%%%%%%%%%%%%%
%%%%%%%%%%%%%%%%%%%%%%%%%%%%%%%%%%%%%%%%%%%%%%%%%%%%%%%%%%%%%%%%%%%%%%%%%%%%%%%%
\end{enumerate}
\bibliography{bibliography}
\end{document}
